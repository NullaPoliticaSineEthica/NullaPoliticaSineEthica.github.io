% ====================================================================
%
% Copyleft 2016-today NPSE.
%
%%%%%%%%%%%%%%%%%%%%%%%%%%%%%%%%%%%%%%%%%%%%%%%%%%%%%%%%%%%%%%%%%%%%%%


\documentclass{beamer}

\usetheme{npse}

\usepackage[spanish]{babel}
\usepackage[utf8]{inputenc}

% Metadata
%%%%%%%%%%%%%%%%%%%%%%%%%%%%%%%%%%%%%%%%%%%%%%%%%%%%%%%%%%%%%%%%%%%%%%

\title{Nulla Politica Sine Ethica}
\subtitle{Principios}
\author{Miembros de NPSE}
\institute[NPSE]{%
  \href{https://www.facebook.com/nullapoliticasineethica}{NullaPoliticaSineEthica}
}
\date[03/2016]{Marzo 2016}
\subject{Nulla Politica Sine Ethica}
\keywords{NPSE, activismo, politica}

% Contents
%%%%%%%%%%%%%%%%%%%%%%%%%%%%%%%%%%%%%%%%%%%%%%%%%%%%%%%%%%%%%%%%%%%%%%

\begin{document}

\begin{frame}[plain]
  \titlepage
\end{frame}

\begin{frame}[plain]{Contenido}
  \tableofcontents[hideallsubsections]
\end{frame}

\section{Diagn\'ostico}

\subsection{Espa\~na en 2016}

\begin{frame}{Espa\~na en 2016}
	\begin{columns}[t]
		\column{.5\textwidth}
		\begin{alertblock}{Una reflexi\'on personal}
      ?‘Qu\'e consideras lo m\'as preocupante?
		\end{alertblock}
		\pause

		\column{.5\textwidth}
		\begin{block}{Encuesta del \href{http://politica.elpais.com/politica/2016/03/08/actualidad/1457427632_719011.html}{CIS}}
			\begin{itemize}
				\item El paro.
				\item La corrupci\'on y el fraude.
				\item La economía.
        \item La política.
        \item Los problemas sociales.
			\end{itemize}
		\end{block}
	\end{columns}
\end{frame}

\begin{frame}{Europa en 2016}
	\begin{columns}[t]
		\column{.5\textwidth}
		\begin{alertblock}{Una reflexi\'on personal}
      ?‘Qu\'e piensas de Europa?
		\end{alertblock}
		\pause

		\column{.5\textwidth}
f		\begin{block}{Preocupaciones recurrentes}
			\begin{itemize}
				\item Falta de democracia.
				\item Tiran\'ia de los mercados.
				\item Supremac\'i{}a del pensamiento \'unico liberal.
        \item Auge de la extrema derecha.
        \item Indiferencia o inoperancia ante crisis sociales.
			\end{itemize}
		\end{block}
	\end{columns}

\end{frame}

\section{Soluciones desencaminadas o insuficientes}

\begin{frame}{?‘Qu\'e se est\'a haciendo?}
  \begin{block}{Manipular}
  \begin{itemize}
    \item La pol\'i{}ica es el nuevo entretenimiento televisivo.
    \item Nos han convencido de que todos los problemas los ha originado una mala gesti\'on de lo p\'ublico.
    \item Se legisla en favor de la minor\'i{}a con mayor poder econ\'omico.
    \item Han conseguido instaurar el miedo en la sociedad.
    \item Los medios de comunicaci\'on sirven siempre, y exclusivamente, a lo que dictan los intereses privados.
  \end{itemize}
  \end{block}
\end{frame}

\begin{frame}{?‘Qu\'e se est\'a haciendo?}
  \begin{block}{Confundir}
  \begin{itemize}
    \item No existe una ciencia econ\'omica con leyes que acatar.
    \item Repetir falsedades para desensibilizarnos.
    \item Hasta lo obvio es interpretable.
    \item El miedo condiciona c\'omo interpretamos los problemas y soluciones.
    \item No nos defendemos, nos dividimos.
  \end{itemize}
  \end{block}
\end{frame}

\section{Lo que hay que hacer}

\subsection{Princicipos fundamentales de NPSE}

\begin{frame}{?Qu\'e propone NPSE?}
  \begin{alertblock}{5 simples medidas}
  \begin{enumerate}
    \item Renta B\'asica Universal Incondicional.
    \item Garantizar derecho a la vivienda digna.
    \item Auditor\'i{a} ciudadana de la deuda.
    \item Interrumpir tratados de libre comercio e intentos de privatizaci\'on.
    \item Erradicar la corrupci\'on.
  \end{enumerate}
  \end{alertblock}
\end{frame}

\subsection{Renta B\'asica Universal Incondicional}

\begin{frame}{Renta B\'asica Universal Incondicional}
  \begin{block}{La medida que lo cambia todo}
  \begin{itemize}
    \item Est\'a acreditada su viabilidad econ\'omica.
    \item La dignidad de cada persona no depende de su empleabilidad.
    \item Proporciona libertad de elecci\'on.
  \end{itemize}
  \end{block}
\end{frame}

\subsection{Derecho a una vivienda digna}

\begin{frame}{Derecho a la vivienda}
  \begin{block}{Hacer efectivo el derecho constitucional}
  \begin{itemize}
    \item Una vivienda es un bien de primera necesidad, no un lujo.
    \item Constituye un importante problema social.
    \item La daci\'on en pago es lo razonable, no lo ut\'opico.
    \item Hay que poner freno a la especulaci\'on.
  \end{itemize}
  \end{block}
\end{frame}

\subsection{Auditor\'i{}a ciudadana de la deuda}

\begin{frame}{Deuda p\'ublica}
  \begin{block}{Analizar la deuda contra\'i{}da}
  \begin{itemize}
    \item Identificar las necesidades reales que se han cubierto.
    \item Separar la especulaci\'on.
    \item Documentar la parte debida a acuerdos contra\'i{}dos sin control democr\'atico.
    \item Consultar a la ciudadan\'ia en cuestiones de especial relevancia.
  \end{itemize}
  \end{block}
\end{frame}

\subsection{Tratados de libre comercio y privatizaciones}

\begin{frame}{Ni tratados ni recortes}
  \begin{block}{TTIP/CETA/TISA}
  \begin{itemize}
    \item Legitiman abusos futuros, consentidos.
    \item Esconden avales p\'ublico sobre beneficios privados de grandes corporaciones.
    \item Hay que recuperar lo perdido con los recortes y las privatizaciones.
    \item Los servicios p\'ublicos son una prioridad.
  \end{itemize}
  \end{block}
\end{frame}

\subsection{Corrupci\'on}

\begin{frame}{La corrupci\'on es erradicable}
  \begin{block}{Proponemos}
  \begin{itemize}
    \item Legislar en contra de las ‘‘puertas giratorias''.
    \item Impedir que corruptos ocupen cargos p\'ublicos.
    \item Definir l\'i{}mites en las remuneraciones p\'ublicas.
    \item Reformar las instituciones ineficientes o innecesarias.
  \end{itemize}
  \end{block}
\end{frame}

\section{Participa y difunde}

\subsection{Combate el miedo}

\begin{frame}{No debemos resignarnos}
  \begin{block}{Hay mucho por hacer}
  \begin{itemize}
    \item Es importante darle la vuelta a la situaci\'on.
    \item La RBUI lo cambia todo.
    \item Los problemas no nos ni mucho menos insalvables.
    \item No seamos v\'i{}ctimas de nuestras opiniones y prejuicios.
    \item Ayuda, comparte, difunde.
  \end{itemize}
  \end{block}
\end{frame}

\subsection{Es urgente}

\begin{frame}{Las prioridades de NPSE son sus 5 principios}
  \begin{block}{Pero hay mucho m\'as por hacer}
  \begin{itemize}
    \item El desempleo tecnol\'ogico y sus consecuencias.
    \item El cambio clim\'atico (deterioro del planeta).
    \item ...
  \end{itemize}
  \end{block}
\end{frame}

\end{document}
